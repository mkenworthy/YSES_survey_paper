%\documentclass[referee]{aa} % for a referee version
\documentclass[onecolumn]{aa} % for a paper on 1 column  
%\documentclass[longauth]{aa} % for the long lists of affiliations 
%\documentclass[letter]{aa} % for the letters 
%\documentclass{aa}
\usepackage{txfonts}
\usepackage{natbib}

\usepackage{graphicx}

\usepackage{color}
\usepackage{hyperref}
\hypersetup{colorlinks=true,allcolors=[rgb]{0,0,0.8}}

% the three lines suppress the hyperref 'link empty' warnings
% explanation at: https://tex.stackexchange.com/questions/345764/journal-class-shows-package-hyperref-warning-suppressing-link-with-empty-targe
\makeatletter
\renewcommand*\aa@pageof{, page \thepage{} of \pageref*{LastPage}}
\makeatother
\usepackage{showyourwork}

\begin{document} 

   \title{The Young Suns Exoplanet Survey}

   \author{M. Kenworthy
          \inst{1}
          \and
          C. Ginski
          \inst{2}
          \and
          Alex Bohn 
          \inst{3}
          \and
          Eric Mamajek
          \inst{4}
          \and 
          N.E.Others
          \inst{4}
          }

   \institute{Leiden Observatory, University of Leiden,
   PO Box 9513, 2300 RA Leiden, The Netherlands\\
   \email{kenworthy@strw.leidenuniv.nl}
         \and
         Zurich address
    \and
    JPL
    \and
    Caltech/IPAC, 1200 E California Blvd, Mail Code 100-22, Pasadena, CA 91125, USA}

   \date{Received XXXX; accepted XXXX}

% \abstract{}{}{}{}{} 
% 5 {} token are mandatory
 
  \abstract
  % context heading (optional)
  % {} leave it empty if necessary  
   {Finding planets is fun.}
  % aims heading (mandatory)
   {We want to determine the gas giant exoplanet formation rate around Sun-like stars.}
  % methods heading (mandatory)
   {Direct imaging in H and K, two epochs on large proper motion stars in the Sco-Cen association, common proper motion identification, follow up observations.}
  % results heading (mandatory)
   {We found at least four planets but at large semimajor axis distances. }
  % conclusions heading (optional), leave it empty if necessary 
   {We can find exoplanets, they are at much larger sma than we expected.}

   \keywords{giant planet formation}

   \maketitle
%
%-------------------------------------------------------------------

   \section{Introduction}

Terrestrial planets are thought to be built up by the quasi-periodic accretion of planetary embryos that generate a significant amount of ejected material.
%
The Earth's moon is believed to have formed from the resulting aftermath of a collision in the early Solar system.
%
Sudden increases of infrared flux from systems known to host debris disks indicate that this is a stochastic process that can occur on timescales of a few months or less.
%

\section{Properties of the star}\label{sec:star}

Gaia EDR3 source 5539970601632026752 at RA=08:15:23.2996, DEC=-38:59:23.304, $d\sim 556$ pc, G=13.4 mag, BP-RP=0.8 mag.

%\begin{figure}
%   \centering
%   \includegraphics[width=\hsize]{figures/asassn-21qj-gmk-sed-fit.png}
%      \caption{VOSA fit to photometry of ASAS-SN J0815 by gmk}
%         \label{fig:sed}
%\end{figure}


%------------------------
\section{Conclusions}\label{sec:conclusion}

   \begin{enumerate}
      \item Lots of planets, quite the surprise...
      \item PLanet occurrence rate of YYYY
   \end{enumerate}

\begin{acknowledgements}
This research made use of Astropy,\footnote{http://www.astropy.org} a community-developed core Python package for Astronomy \citep{astropy:2013, astropy:2018}, Python \citep{vanRossum95,Oliphant07}, Matplotlib \citep{Hunter07}, numpy \citep{Oliphant06,vanderWalt11} and SciPy \citep{Virtanen20}.
\end{acknowledgements}

\bibliographystyle{aa}
\bibliography{bib}

\end{document}
