%\documentclass[referee]{aa} % for a referee version
\documentclass[onecolumn]{aa} % for a paper on 1 column  
%\documentclass[longauth]{aa} % for the long lists of affiliations 
%\documentclass[letter]{aa} % for the letters 
%\documentclass{aa}
\usepackage{txfonts}
\usepackage{natbib}
\usepackage{lscape} % for the rotated longtable
\usepackage{longtable}
\usepackage{graphicx}

\usepackage{lipsum}  

\usepackage{color}
\usepackage{hyperref}
\hypersetup{colorlinks=true,allcolors=[rgb]{0,0,0.8}}

% the three lines suppress the hyperref 'link empty' warnings
% explanation at: https://tex.stackexchange.com/questions/345764/journal-class-shows-package-hyperref-warning-suppressing-link-with-empty-targe
\makeatletter
\renewcommand*\aa@pageof{, page \thepage{} of \pageref*{LastPage}}
\makeatother
\usepackage{showyourwork}

\begin{document} 

   \title{The Young Suns Exoplanet Survey}

   \author{M. Kenworthy
          \inst{1}
          \and
          C. Ginski
          \inst{2}
          \and
          Alex Bohn 
          \inst{3}
          \and
          Eric Mamajek
          \inst{4}
          \and 
          N.E.Others
          \inst{4}
          }

   \institute{Leiden Observatory, University of Leiden,
   PO Box 9513, 2300 RA Leiden, The Netherlands\\
   \email{kenworthy@strw.leidenuniv.nl}
         \and
         Zurich address
    \and
    JPL
    \and
    Caltech/IPAC, 1200 E California Blvd, Mail Code 100-22, Pasadena, CA 91125, USA}

   \date{Received XXXX; accepted XXXX}

% \abstract{}{}{}{}{} 
% 5 {} token are mandatory
 
  \abstract
  % context heading (optional)
  % {} leave it empty if necessary  
   {Finding planets is fun.}
  % aims heading (mandatory)
   {We want to determine the gas giant exoplanet formation rate around Sun-like stars.}
  % methods heading (mandatory)
   {Direct imaging in H and K, two epochs on large proper motion stars in the Sco-Cen association, common proper motion identification, follow up observations.}
  % results heading (mandatory)
   {We found at least four planets but at large semimajor axis distances. }
  % conclusions heading (optional), leave it empty if necessary 
   {We can find exoplanets, they are at much larger sma than we expected.}

   \keywords{giant planet formation}

   \maketitle
%
%-------------------------------------------------------------------

\section{Introduction}

Despite several remarkable exoplanet and brown dwarf discoveries by high-contrast imaging at high angular resolution in the past few years (e.g., Marois et al. 2008, 2010; Schmidt et al. 2008; Lagrange et al. 2010; Rameau et al. 2013; Bailey et al. 2014; Macintosh et al. 2015; Chauvin et al. 2017; Keppler et al. 2018; Haffert et al. 2019; Janson et al. 2019; Bohn et al. 2020a,b)
there is an ongoing debate regarding the formation mechanisms that create these super-Jovian gas giants with semimajor axes greater than 10 au. It is unclear whether these companions have a star-like origin from a collapsing molecular cloud that is bro- ken up into fragments, creating planetary-mass objects similar to a stellar binary (Kroupa 2001; Chabrier 2003), or through for- mation in a circumstellar disk instead. The classical bottom-up framework postulates formation via core accretion by

\section{Survey description}\label{sec:sample}

SPHERE description.

Selecting stars from Pecaut XXXX and selecting for mass, distance and proper motion.

Strategy was two images in two filters with IRDIS, wait for a while, then take another image later on.

\lipsum[2-8]

\section{Observations and data reduction}\label{sec:obs}
\lipsum[2-8]

\subsection{Using RDI and PCA for halo suppression}
\lipsum[2-8]

\subsection{Images and detected candidate companions}
\lipsum[2-8]

\section{Detected circumstellar companions}\label{sec:companions}
\lipsum[2-8]

TABLE XXXX Detected companion properties

\lipsum[2-8]

\section{Discussion}\label{sec:discuss}
\lipsum[2-8]

\section{Conclusions}\label{sec:conc}
\lipsum[2-8]

%\begin{figure}
%   \centering
%   \includegraphics[width=\hsize]{figures/asassn-21qj-gmk-sed-fit.png}
%      \caption{VOSA fit to photometry of ASAS-SN J0815 by gmk}
%         \label{fig:sed}
%\end{figure}


%------------------------
\section{Conclusions}\label{sec:conclusion}

   \begin{enumerate}
      \item Lots of planets, quite the surprise...
      \item PLanet occurrence rate of YYYY
   \end{enumerate}

\begin{acknowledgements}
This research made use of Astropy,\footnote{http://www.astropy.org} a community-developed core Python package for Astronomy \citep{astropy:2013, astropy:2018}, Python \citep{vanRossum95,Oliphant07}, Matplotlib \citep{Hunter07}, numpy \citep{Oliphant06,vanderWalt11} and SciPy \citep{Virtanen20}.
\end{acknowledgements}

\bibliographystyle{aa}
\bibliography{bib}


\longtab{
\begin{landscape}
\begin{longtable}{cccccccccccc}
\caption{\label{tab:targets}Star targets and their properties}\\
\hline\hline
Target & RA & Dec & Distance & Mass & V & K & Observation date 1 & NEXP NDIT NDIT & $\left < \omega \right >$ & X & $\tau_0$ \\
\hline
\endfirsthead
\caption{continued.}\\
\hline\hline
Target & RA & Dec & Distance & Mass & V & K & Observation date 1 & NEXP NDIT NDIT & $\left < \omega \right >$ & X & $\tau_0$ \\
\hline
\endhead
\hline
\endfoot
J11272881-3952572 & 01:23:45 & -33:22:10 & 110 & 1.1 & 9 & 8 & 2017-04-18 & 4x1x32 & 1.51 & 1.1 & 1.4 \\
J11320835-5803199 & 16:19:59 & -32:12:12 & 125 & 0.9 & 10 & 7 & 2017-06-17 & 4x1x32 & 0.67 & 1.47 & 2.9 \\
J11272881-3952572 & 01:23:45 & -33:22:10 & 110 & 1.1 & 9 & 8 & 2017-04-18 & 4x1x32 & 1.51 & 1.1 & 1.4 \\
J11320835-5803199 & 16:19:59 & -32:12:12 & 125 & 0.9 & 10 & 7 & 2017-06-17 & 4x1x32 & 0.67 & 1.47 & 2.9 \\
J11272881-3952572 & 01:23:45 & -33:22:10 & 110 & 1.1 & 9 & 8 & 2017-04-18 & 4x1x32 & 1.51 & 1.1 & 1.4 \\
J11320835-5803199 & 16:19:59 & -32:12:12 & 125 & 0.9 & 10 & 7 & 2017-06-17 & 4x1x32 & 0.67 & 1.47 & 2.9 \\
J11272881-3952572 & 01:23:45 & -33:22:10 & 110 & 1.1 & 9 & 8 & 2017-04-18 & 4x1x32 & 1.51 & 1.1 & 1.4 \\
J11320835-5803199 & 16:19:59 & -32:12:12 & 125 & 0.9 & 10 & 7 & 2017-06-17 & 4x1x32 & 0.67 & 1.47 & 2.9 \\
J11272881-3952572 & 01:23:45 & -33:22:10 & 110 & 1.1 & 9 & 8 & 2017-04-18 & 4x1x32 & 1.51 & 1.1 & 1.4 \\
J11320835-5803199 & 16:19:59 & -32:12:12 & 125 & 0.9 & 10 & 7 & 2017-06-17 & 4x1x32 & 0.67 & 1.47 & 2.9 \\
J11272881-3952572 & 01:23:45 & -33:22:10 & 110 & 1.1 & 9 & 8 & 2017-04-18 & 4x1x32 & 1.51 & 1.1 & 1.4 \\
J11320835-5803199 & 16:19:59 & -32:12:12 & 125 & 0.9 & 10 & 7 & 2017-06-17 & 4x1x32 & 0.67 & 1.47 & 2.9 \\
J11272881-3952572 & 01:23:45 & -33:22:10 & 110 & 1.1 & 9 & 8 & 2017-04-18 & 4x1x32 & 1.51 & 1.1 & 1.4 \\
J11320835-5803199 & 16:19:59 & -32:12:12 & 125 & 0.9 & 10 & 7 & 2017-06-17 & 4x1x32 & 0.67 & 1.47 & 2.9 \\
J11320835-5803199 & 16:19:59 & -32:12:12 & 125 & 0.9 & 10 & 7 & 2017-06-17 & 4x1x32 & 0.67 & 1.47 & 2.9 \\
J11272881-3952572 & 01:23:45 & -33:22:10 & 110 & 1.1 & 9 & 8 & 2017-04-18 & 4x1x32 & 1.51 & 1.1 & 1.4 \\
J11320835-5803199 & 16:19:59 & -32:12:12 & 125 & 0.9 & 10 & 7 & 2017-06-17 & 4x1x32 & 0.67 & 1.47 & 2.9 \\
J11272881-3952572 & 01:23:45 & -33:22:10 & 110 & 1.1 & 9 & 8 & 2017-04-18 & 4x1x32 & 1.51 & 1.1 & 1.4 \\
J11320835-5803199 & 16:19:59 & -32:12:12 & 125 & 0.9 & 10 & 7 & 2017-06-17 & 4x1x32 & 0.67 & 1.47 & 2.9 \\
J11272881-3952572 & 01:23:45 & -33:22:10 & 110 & 1.1 & 9 & 8 & 2017-04-18 & 4x1x32 & 1.51 & 1.1 & 1.4 \\
J11320835-5803199 & 16:19:59 & -32:12:12 & 125 & 0.9 & 10 & 7 & 2017-06-17 & 4x1x32 & 0.67 & 1.47 & 2.9 \\
J11272881-3952572 & 01:23:45 & -33:22:10 & 110 & 1.1 & 9 & 8 & 2017-04-18 & 4x1x32 & 1.51 & 1.1 & 1.4 \\
J11320835-5803199 & 16:19:59 & -32:12:12 & 125 & 0.9 & 10 & 7 & 2017-06-17 & 4x1x32 & 0.67 & 1.47 & 2.9 \\
J11272881-3952572 & 01:23:45 & -33:22:10 & 110 & 1.1 & 9 & 8 & 2017-04-18 & 4x1x32 & 1.51 & 1.1 & 1.4 \\
J11320835-5803199 & 16:19:59 & -32:12:12 & 125 & 0.9 & 10 & 7 & 2017-06-17 & 4x1x32 & 0.67 & 1.47 & 2.9 \\
J11320835-5803199 & 16:19:59 & -32:12:12 & 125 & 0.9 & 10 & 7 & 2017-06-17 & 4x1x32 & 0.67 & 1.47 & 2.9 \\
J11272881-3952572 & 01:23:45 & -33:22:10 & 110 & 1.1 & 9 & 8 & 2017-04-18 & 4x1x32 & 1.51 & 1.1 & 1.4 \\
J11320835-5803199 & 16:19:59 & -32:12:12 & 125 & 0.9 & 10 & 7 & 2017-06-17 & 4x1x32 & 0.67 & 1.47 & 2.9 \\
J11272881-3952572 & 01:23:45 & -33:22:10 & 110 & 1.1 & 9 & 8 & 2017-04-18 & 4x1x32 & 1.51 & 1.1 & 1.4 \\
J11320835-5803199 & 16:19:59 & -32:12:12 & 125 & 0.9 & 10 & 7 & 2017-06-17 & 4x1x32 & 0.67 & 1.47 & 2.9 \\
J11272881-3952572 & 01:23:45 & -33:22:10 & 110 & 1.1 & 9 & 8 & 2017-04-18 & 4x1x32 & 1.51 & 1.1 & 1.4 \\
J11320835-5803199 & 16:19:59 & -32:12:12 & 125 & 0.9 & 10 & 7 & 2017-06-17 & 4x1x32 & 0.67 & 1.47 & 2.9 \\
J11272881-3952572 & 01:23:45 & -33:22:10 & 110 & 1.1 & 9 & 8 & 2017-04-18 & 4x1x32 & 1.51 & 1.1 & 1.4 \\
J11320835-5803199 & 16:19:59 & -32:12:12 & 125 & 0.9 & 10 & 7 & 2017-06-17 & 4x1x32 & 0.67 & 1.47 & 2.9 \\
J11272881-3952572 & 01:23:45 & -33:22:10 & 110 & 1.1 & 9 & 8 & 2017-04-18 & 4x1x32 & 1.51 & 1.1 & 1.4 \\
J11320835-5803199 & 16:19:59 & -32:12:12 & 125 & 0.9 & 10 & 7 & 2017-06-17 & 4x1x32 & 0.67 & 1.47 & 2.9 \\

\end{longtable}
\end{landscape}
}



\end{document}
